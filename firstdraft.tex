\documentclass[12pt]{article}

\DeclareMathSizes{12}{30}{16}{12}
\usepackage{a4wide}
\usepackage{caption} 
\captionsetup[table]{skip=10pt}
\usepackage[utf8]{inputenc}
\usepackage{graphicx}
\usepackage{biblatex}
\addbibresource{references.bib}
\usepackage{setspace}

\title{The Unjustified Rise of College Tuition Over the Years}
\author{Joshua Rabara }
\date{November 14 2022}

\begin{document}
\maketitle

\doublespacing

\begin{abstract}

Within this paper, there contains an in depth analysis on the going rise of college tuition throughout the nation and as to whether or not such high inflation was justified. This was conducted by first establishing that college tuition in both private and public universities has risen significantly within the past four decades. Then, there was research done to find out whether not college attendance has risen as well over the same amount of time. After all that, linear regression models were developed to generate proper mathematical outcomes. In terms of whether or not there even is correlation in years progressing and that of both tuition prices and attendance. From all this, it was found that all dependent variables in question did have a steep escalation over the past four decades. In particular, the rate of college cost inflation. Due to the fact that over the years, cost of attending college did not increase in relativity to that of the national inflation rate, it could be determined that there were far greater issues that were plaguing students with such financial burden that could be better controlled by universities themselves. By doing all of this, it can be proven that rather than economic or political aspects, college tuition is unjustifiably risen for the sake of much more selfish intentions acted upon by universities themselves. Establishing that such institutions themselves are to blame for the unwarranted upsurge in today's financially draining college market.

\end{abstract}

\section{Introduction}

When looking at the current state of college tuition, anything but drops in cost occur. Why must college be priced to the point where the government must intervene and host unresolved debates about such a topic? Over the past few decades, there has been a 326\% increase in college tuition from 1987 to 2007 (Best \& Keppo, 2011). Besides the continual growth of inflation within the country, universities' only viable reason for increasing tuition prices would solely be to maximize profit from the financially struggling young Americans that desperately need college degrees to land any job in today's society. 

Not only are tuition prices unbelievable priced, but such high costs then put thousands of college students in huge financial debt. Asking teenagers who had just graduated high school to then apply for thousands of dollars in student loans of which will be building interest. Thus leading them to pay off immeasurable amounts of debt the second they are to graduate. Along with that, college education has not changed much in the past 20 to 30 years in terms of quality. Tenure professors tend to not change their curriculum for the classes they regularly teach. Along with that, the continued practice of in person teaching continues to be the more expensive educational option (McArdle, 2012). Thus resulting in the question as to why have prices skyrocketed. Maybe attendances have risen but even then, the only qualifications that appeal to those attending higher education is the price and how prestigious of an institution it is (Nordvall, 2016). 

In a more broad sense, it cannot be pinpointed as to how rising college tuition benefits anybody besides the institutions that cause such financial difficulty. Prior research typically describes data of increasing tuition costs and answers the question of merely how much has tuition increased. However through this analysis, there will be data provided to see if college attendance has changed, how tuition has overall risen throughout the country over the years, and applications from a business and ethical standpoint to determine whether or not colleges are justified in becoming so financially demanding on those they are supposed to benefit. Simulations such as tables and plots will provide visual representations of data to such topics to better illustrate the changes over time. Methods and mathematical applications will describe such graphs and provide proof as to how we can make such assumptions. Finally, a discussion portion can finally establish if college tuition rates are justified, or if universities over the years have been raising tuition to generate more profit from generations of people that seek more opportunities in life through the only way they know how. 


\section{Methods}
For data, there will be analyses on various data provided by sources from Taylor \& Francis and the UCONN Library databases. Regression models will be generated in order to visually represent the trends that are discovered while also being used to predict future values for subjects such as college tuition or attendances by the slope of each model. Given by the equations: 

\begin{equation}
\hat{Y}_i = \hat{\beta}_0 + \hat{\beta}_1 X_i + \hat{\epsilon}_i
\end{equation}

\begin{equation}
\hat{\beta}_1 = \frac{\sum(X_i – \bar{X}) (Y_i – \bar{Y})} {\sum(X_i – \bar{X})^2}
\end{equation}

A correlation coefficient will be added to further prove if the data has a strong correlation. With this statistical proof, we will be able to measure the strength of the linear relationship between variables if there is one present. A correlation coefficient greater than 0 is a positive linear relationship, and less than 0 will result in a negative relationship. 

\begin{equation}
\ {r} = \frac{{}\sum_{i=1}^{n} (x_i - \overline{x})(y_i - \overline{y})}
{\sqrt{\sum_{i=1}^{n} (x_i - \overline{x})^2(y_i - \overline{y})^2}}
\end{equation}

Finally, a hypothesis test in the form of a chi-square test will be used. The purpose of the hypothesis test will be to confirm if the beta coefficients hold actual significance to the linear regression model. In the presence of a chi-square test, there is analysis to find a significant relationship between variables and to see if the sample properly represents the population. (Khandelwal, 2021). 

\begin{equation}
\[\tilde{\chi}^2=\frac{1}{d}\sum_{k=1}^{n} \frac{(O_k - E_k)^2}{E_k}\]
\end{equation}

\section{Simulation}

The process begins by first collecting data on the average college tuition for every year from 1970 to 2019 as shown below. This data compares the national average tuition costs between both public and private institutions (Education Data Initiative, 2022). Then from there, the total attendance of both public and private 4 year universities was acquired by the same source in order to develop a sense of whether or not demand for higher education has either dropped or increased. To undergo the mathematical simulation, all data had to be put into R Studio and created into data frames. From there, it was possible to obtain summary statistics and generate graphs. Multiple versions of code were also run to generate slope equations for each category observed. 

\begin{table}
\caption{Average Tuition Over the Years Between Private and Public Colleges}
\begin{minipage}{.5\linewidth}
\centering
\begin{tabular}{l|c r}
\hline
Year & Public 4-Year & Private 4-Year\\
\hline
2019	& 9,349        & 32,769\\
\hline
2018	& 9,212        & 31,883\\
\hline
2017	& 9,036        & 30,723\\
\hline
2016	& 8,804        & 29,476\\
\hline
2015	& 8,778        & 27,942\\
\hline
2014	& 8,543        & 26,739\\
\hline
2013	& 8,312        & 25,707\\
\hline
2012	& 8,070        & 24,523\\
\hline
2011	& 7,713        & 23,464\\
\hline
2010	& 7,132        & 22,677\\
\hline
2009	& 6,717        & 22,269\\
\hline
2008	& 6,312        & 22,040\\
\hline
2007	& 5,943        & 21,427\\
\hline
2006	& 5,666        & 20,517\\
\hline
2005	& 5,351        & 19,292\\
\hline
2004	& 5,027        & 18,604\\
\hline
2003	& 4,587        & 17,763\\
\hline
2002	& 4,046        & 16,826\\
\hline
2001	& 3,735        & 16,211\\
\hline
2000	& 3,501        & 15,470\\
\hline
1999    & 3,349	       & 14,616\\
\hline
1998	& 3,229	       & 13,973\\
\hline
1997	& 3,110	       & 13,344\\
\hline
1996	& 2,987	       & 12,881\\
\hline
1995	& 2,848	       & 12,243\\
\hline
1994	& 2,681	       & 11,481\\
\hline
1993	& 2,537	       & 10,952\\
\hline
1992	& 2,349	       & 10,294\\
\hline
\end{tabular}
\end{minipage}%
\begin{minipage}{.5\linewidth}
\centering
\begin{tabular}{l| c r }
\hline
Year & Public 4-Year & Private 4-Year\\
\hline
1991	& 2,117	       & 9,759\\
\hline
1990	& 1,888	       & 9,083\\
\hline
1989	& 1,780	       & 8,396\\
\hline
1988	& 1,646	       & 7,722\\
\hline
1987	& 1,537	       & 7,116\\
\hline
1986	& 1,414	       & 6,658\\
\hline
1985	& 1,318	       & 6,121\\
\hline
1984	& 1,228	       & 5,556\\
\hline
1983	& 1,148	       & 5,093\\
\hline
1982	& 1,031	       & 4,639\\
\hline
1981	& 909	       & 4,113\\
\hline
1980	& 804	       & 3,617\\
\hline
1979	& 738	       & 3,225\\
\hline
1978	& 688	       & 2,958\\
\hline
1977	& 655	       & 2,700\\
\hline
1976	& 617	       & 2,534\\
\hline
1975	& 542	       & 2,291\\
\hline
1974	& 512	       & 2,130\\
\hline
1973	& 514	       & 2,045\\
\hline
1972	& 503	       & 1,948\\
\hline
1971	& 428	       & 1,832\\
\hline 
1970    & 394          & 1706\\
\hline
\end{tabular}
\end{minipage} 
\end{table}

\begin{figure}[htp]
    \centering
    \includegraphics[width=10cm]{Public Tuition Plot.pdf}
    \caption{A linear model of the given data for Public College tuition
             since 1970}
\end{figure}

\begin{figure}[htp]
    \centering
    \includegraphics[width=10cm]{Screen Shot 2022-11-09 at 7.13.11 PM.png}
    \caption{Descriptive statistics marking the slope at 196 for every unit increase with a y-intercept of -387,400}
\end{figure}

\begin{figure}[htp]
    \centering
    \includegraphics[width=10cm]{Private College Tuition.pdf}
    \caption{A linear model of the given data for Private College tuition}
\end{figure}

\begin{figure}[htp]
    \centering
    \includegraphics[width=10cm]{Screen Shot 2022-11-09 at 7.15.06 PM.png}
    \caption{Descriptive statistics marking the slope at 637.2 for every unit increase with a y-intercept of -1,258,000 }
    \label{fig:galaxy}
\end{figure}


\begin{table}
\caption{Total College Enrollment since 1970}
\begin{minipage}{.5\linewidth}
\centering
\begin{tabular}{l| c r}
\hline
Year    & Total College Enrollment\\
\hline
2019	&19,630,178\\
\hline
2018	& 19,651,412\\	
\hline
2017	& 19,778,151\\	
\hline
2016	& 19,846,904\\	
\hline
2015	& 19,988,204\\	
\hline
2014	& 20,209,092\\	
\hline
2013	& 20,376,677\\	
\hline
2012	& 20,644,478\\	
\hline
2011	& 21,010,590\\
\hline
2010	& 21,019,438\\	
\hline
2009	& 20,313,594\\	
\hline
2008	& 19,081,686\\
\hline
2007	& 18,258,138\\
\hline
2006	& 17,754,230\\
\hline
2005	& 17,487,475\\
\hline
2004	& 17,272,044\\
\hline
2003	& 16,911,481\\
\hline
2002	& 16,611,711\\
\hline
2001	& 15,927,987\\	
\hline
2000	& 15,312,289\\	
\hline
1999	& 14,849,691\\
\hline
1998	& 14,506,967\\
\hline
1997	& 14,502,334\\
\hline
1996	& 14,367,520\\
\hline
1995	& 14,261,781\\
\hline
1994	& 14,278,790\\
\hline
1993	& 14,304,803\\
\hline
1992	& 14,487,359\\
\hline
\end{tabular}
\end{minipage}%
\begin{minipage}{.5\linewidth}
\centering
\begin{tabular}{l| c r}
\hline
Year    & Total College Enrollment\\
\hline
1991	& 14,358,953\\	
\hline
1990	& 13,818,637\\	
\hline
1989	& 13,538,560\\	
\hline
1988	& 13,055,337\\	
\hline
1987	& 12,766,642\\
\hline
1986	& 12,503,511\\	
\hline
1985	& 12,247,055\\	
\hline
1984	& 12,241,940\\	
\hline
1983	& 12,464,661\\	
\hline
1982	& 12,425,780\\
\hline
1981	& 12,371,672\\
\hline
1980	& 12,096,895\\
\hline
1979	& 11,569,899\\	
\hline
1978	& 11,260,092\\
\hline
1977	& 11,285,787\\	
\hline
1976	& 11,012,137\\	
\hline
1975	& 11,184,859\\	
\hline
1974	& 10,223,729\\
\hline
1973	& 9,602,123\\
\hline
1972	& 9,214,860\\
\hline
1971	& 8,948,644\\
\hline
1970	& 8,580,887\\
\hline
\end{tabular}
\end{minipage} 
\end{table}

\begin{figure}[htp]
    \centering
    \includegraphics[width=10cm]{Attendance Plot.png}
    \caption{A linear model of the given data for total 4 year college attendance
             since 1970}
    \label{fig:galaxy}
\end{figure}

\begin{figure}[htp]
    \centering
    \includegraphics[width=10cm]{Screen Shot 2022-11-09 at 8.50.30 PM.png}
    \caption{Descriptive statistics marking the slope at 244,600 for every unit increase with a y-intercept of -472,900,000}
    \label{fig:galaxy}
\end{figure}

\section{Application}
From the given graphs and data presented, we can see that all of the data has a positive slope from 1970 to 2019. From an attendance standpoint, the amount of people enrolled in four year universities increases almost every year. We can credit this to how much of a demand that college education has become in the always evolving job market. With the better and high paying jobs that having a college degree can lead one to, it can be observed that this also benefits the individual in other ways. Correlations have been found that illustrate how those with degrees further experience higher standards of living and better social/personal well being (Chan, 2016). With promises of better quality of life, it's understandable how correlation grew strong between college attendance as time passed. Due to high attendance, universities should have no difficulty in raising funds to support their influx of students. Therefore, how is it that prices have been soaring to margins that are vastly unobtainable to the average student. So unobtainable in fact, many students have been forced to put loans under their names just for higher education and more financial opportunities. Universities obviously don't care how much turmoil they put recent high school graduates under by asking for so much money, as shown mathematically, they have no sympathy for those that cannot afford their services, as long as they're able to get students that can afford it. 

When looking at both progressing and current college trends, average tuition across the country did not drop once over the span of 30 plus years. The ailment of inflation can be easily ruled out due to the fact that since 1970, the cumulative inflation rate has been around roughly 668\%. However, compared to the acquired data, the inflation of college tuition is nearly 2,000\% according to the research conducted. It can be seen that non-profit universities aim to drive up their revenue by spending all the money they are able to earn. This can be in various ways by installing such aspects that improve quality of life on campus, or fund aspects of the school that make it's image better and more well known. For instance, in such cases where schools will pour money into their award winning sports teams rather than improve the overall education in which most students are there for. This form of marketing and setting an image for a university can both cost a lot, but also draw many new faces. From this, their new enrollments can see higher costs as the university attempts to keep up the facade of being prestigious due to aspects that don't help the student. Marking the fact that cost control is derived from revenue control in this example. 

Another factor in rising tuition can be the result of a phenomenon that only affects businesses that provide services known as cost disease. The term in use is asserted as how the cost of service based businesses are condemned to rise at a rate significantly greater than the economy's rate of inflation (Baumol, 2012). This typically occurs due to that fact that service companies cannot do a lot to be more productive without having it seem as though corners are being cut. Therefore, in order to both increase productivity while displaying the facade that quality is being kept the same, universities must charge more money. However, this does not guarantee that quality stays at such a high degree. In cost disease, the cost control theoretically leads to quality deterioration. That being said, in terms of universities, more students must be accepted as data has shown. However, with more students means additional issues that must be tackled. For instance more classes that instructors must teach will lead to higher productivity levels, but more classes does not necessarily mean higher quality of education (Archibald, 2008). Such a practice being dragged out for a multitude of years poses the struggle of productivity not being a measure of proper performance. Sacrificing students' education just to increase productivity and cost is an injustice to those who are forced to participate in a social practice that is a basic requirement for any sort of success in the future. 

\section{Discussion}
Given the following data and resources provided, it's evident that besides a mathematical standpoint, the reasons towards tuition expansion is both note able and unwarranted. From research provided, it's determined that such rises in college tuition has never been justified throughout history in terms of benefiting students from a business or ethical standpoint. The tuition inflation rate has never matched that of the country's, making such an argument invalid. After this, its proven that prices are simply raised due to the idea of prestige and popularity. Such funds are spent on frivolous ventures rather than to benefit students from an educational standpoint. In order to see a decline in tuition, there would need to be an end to the desire for popularity. It seems preposterous that institutions care significantly more about their names being known rather than the education benefits that they're able to provide to the individual. Or, if such a route is undesirable, there is also the option of fueling more education and well being based programs in colleges to make such financially draining demands more reasonable. 


\section{Appendix}

All data used in the this paper can be credited to the Education Data Initiative. A team of researchers that are dedicated in collecting statistics on the US education system in the hopes of making such data more accessible to the public. All the code used for the simulations were done in R Studio. All references were sourced from Google Scholar. To guarantee college inflation rates actually did raise to around 2000\%, other sources were researched to ensure this values impact and truth. To avoid the error of putting data into R individually, all values were copy and pasted into R to establish no human error when it came to calculations.

\begin{figure}[htp]
    \centering
    \includegraphics[width=10cm]{Screen Shot 2022-11-14 at 4.07.53 PM.png}
    \caption{An image of code used for running mathematical simulations}
    \label{fig:galaxy}
\end{figure}

\begin{figure}[htp]
    \centering
    \includegraphics[width=10cm]{Screen Shot 2022-11-14 at 1.21.19 PM.png}
    \caption{An image of code used for running mathematical simulations}
    \label{fig:galaxy}
\end{figure}

\begin{figure}[htp]
    \centering
    \includegraphics[width=10cm]{Screen Shot 2022-11-14 at 1.41.14 PM.png}
    \caption{An image of code used for running mathematical simulations}
    \label{fig:galaxy}
\end{figure}

\begin{figure}[htp]
    \centering
    \includegraphics[width=10cm]{Screen Shot 2022-11-14 at 1.44.13 PM.png}
    \caption{An image of code used for running mathematical simulations}
    \label{fig:galaxy}
\end{figure}

\section{References}
Please refer to references.bib for the time being.

\end{document}
