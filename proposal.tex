\documentclass[12pt]{article}

%% preamble: Keep it clean; only include those you need
\usepackage{amsmath}
\usepackage[margin = 1in]{geometry}
\usepackage{graphicx}
\usepackage{booktabs}
\usepackage{natbib}
\usepackage{epstopdf}
\epstopdfsetup{update}
\usepackage{grfext}

% for space filling
% highlighting hyper links
\usepackage[colorlinks=true, citecolor=blue]{hyperref}


%% meta data

\title{Proposal: Current Average American Income Spending Power Compared to Prior Spending Power Before Inflation }
\author{Joshua Rabara\\
  Department of Statistics\\
  University of Connecticut
}

\begin{document}
\maketitle


\paragraph{Introduction}
With headlines of the country on the brink of another recession, it only makes sense to look at the past to help fix the present. When analyzing current economic trends in the nation, it can be seen that the spending power of the American dollar has drastically dropped throughout not only years, but generations. If there were to be a comparison of financial obstacles, such as college tuition prices between now and the 1960's, it would be seen that there are significant cost differences after adjusting for inflation. Prior to inflation, the average college tuition would be proportionate to about one fifth of the average salary. Now when doing the same comparison, you can see that the cost would be almost half of the current national salary. From this comparison, this shows that not only are more financially risky goals becoming more unobtainable for the average person in America, but cost to income ratios have become greatly disproportionate as time has passed.

\paragraph{Specific Aims}
From this research, the goal is to highlight the negative effect of inflation on the current economic status of the country. Along with that is to demonstrate how if this trend were to continue, it would only get worse for the average American if a change is not made. 

\paragraph{Data}
The data intended for use would be a collection of both past and present comparisons between salaries, various prices that have drastically fluctuated, and increasing inflation rates throughout the years. We can obtain such data from sources such as the US Census Bureau and the Bureau of Economic Analysis under the the US Department of Commerce.

\paragraph{Research Design and Methods}
A lot of the design and methodology from the research being conducted would be a lot of quantitative data analysis. Comparisons between the both past and present would be used to show any significant differences between the time periods. A lot of data would be obtained from sources that can describe relative average values for years prior. Giving this research a wide time range to work with which can show changes over the course of time. With the use of linear regression to show rise in inflation over the years given by:

\[
  Y = \beta_{0} + \beta_{1}X + E
\]



\paragraph{Discussion}
With this research, it can highlight the proportionate differences in what other generations have gone through financially. Such research can help people of different times to better understand as to why it's far more difficult for younger generations to become financially independent. The differences in economies over time cause a change in both priorities and  mindsets that definitely give better insights to younger mindsets. When such information is presented, it can be more commonly known that with more controlled and lowered inflation rates, means a more thriving and lucrative economy. Not just inflation rates have a chance to change but the possibility of more adjusted minimum wages in order to be more sufficient in times of such economic turmoil.

\paragraph{Conclusion}
When looking at the state of the American economy today, one can notice that prices of today far exceed what the average income can afford. When writing this in a collegiate study, one can relate this research to the case of why higher education was far more obtainable then versus the present. Quality of academics haven't changed much over the years, but the quality of the economy has. Using comparisons of inflation rates over the years can give insights to both average and cumulative inflation rates over the course of decades. Giving a better understanding as to the current state of our economy and how it far differs to years prior. From this research when we see the change in the value of US currency, the goal is to have wages and salaries be adjusted for more suitable living conditions and experiences. 


\end{document}
